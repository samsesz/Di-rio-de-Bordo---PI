%%%% fatec-article.tex, 2024/03/10

%% Classe de documento
\documentclass[
landscape,
  a4paper,%% Tamanho de papel: a4paper, letterpaper (^), etc.
  12pt,%% Tamanho de fonte: 10pt (^), 11pt, 12pt, etc.
  english,%% Idioma secundário (penúltimo) (>)
  brazilian,%% Idioma primário (último) (>)
]{article}

%% Pacotes utilizados
\usepackage[]{fatec-article}
\usepackage{setspace}

%% Processamento de entradas (itens) do índice remissivo (makeindex)
%\makeindex%

%% Arquivo(s) de referências
%\addbibresource{fatec-article.bib}

%% Início do documento
\begin{document}

% Seções e subseções
%\section{Título de Seção Primária}%

%\subsection{Título de Seção Secundária}%

%\subsubsection{Título de Seção Terciária}%

%\paragraph{Título de seção quaternária}%

%\subparagraph{Título de seção quinária}%

%\section*{Diário de Bordo}%
\section*{Instruções para o preenchimento}
\doublespacing
\begin{enumerate}
    \item O Diário de Bordo é usado para registrar atividades, progressos, ideias e desafios enfrentados em um projeto ou durante a rotina de trabalho. Serve como um registro cronológico e detalhado das operações diárias, facilitando a organização e o acompanhamento das tarefas.
    \doublespacing
    \item Durante o registro das atividades deve-se incluir detalhes como datas, horários, descrições de tarefas, nomes de participantes e observações relevantes.  Esta documentação contínua ajuda na avaliação do progresso de projetos ou atividades, permitindo ajustes e melhorias contínuas nos processos.
    \doublespacing
    \item Para evidenciar a realização das tarefas, você poderá utilizar a criação de anexos para adicionar anotações, fotos, prints, questionários, entre outros.
\end{enumerate}

\break

 \begin{table}[]
\centering
\begin{tabular}{|l|l|l|l|l|}
\hline
Nome da Atividade & Data de início & Data de término & Responsável pela atividade & Descrição da atividade realizada \\ \hline
Reunião em grupo para discussão da distribuição das tarefas& 13/09/2024& 13/09/2024&Todos os membros   &Discussão com os membros para melhor entendimento das tarefas a serem feitas\\ \hline
Inicio do figma& 14/09/2024&10/11/2024& Marcelo Augusto&Criação do figma com as ideias discutidas em grupo\\ \hline
Criação da logo do projeto e da equipe&15/09/2024&15/09/2024&Samia Muniz e Marcelo Augusto& Discussões criativas sobre como seria a logo do projeto\\ \hline
Inicio do desenvolvimento do site&20/09/2024&14/11/2024&Marcelo Augusto e Samia Muniz&Discussões sobre a montagem do site e o design do mesmo\\ \hline
Inicio do desenvolvimento dos diagramas de banco de dados&18/09/2024&14/11/2024&Catarine Pereira&Criação e modelagem do modelo conceitual e logico do banco de dados do projeto\\ \hline
Inicio do desenvolvimento dos diagramas de engenharia&19/09/2024&14/11/2024&Samia Muniz&Criação do diagrama de caso de uso e do canvas do aplicativo\\ \hline
Inicio do desenvolvimento do diagrama de rede&21/09/2024&14/11/2024&Catarine Pereira&Criação do diagrama de redes do aplicativo com orientação do professor\\ \hline
Inicio do artigo&21/09/2024&14/11/2024&Samia Muniz&Criação do artigo e pesquisas relacionadas ao assunto do projeto\\ \hline
Modificações no figma&09/11/2024&14/11/2024&Marcelo Augusto e Catarine Pereira&Modificações de cores e funcionalidades do figma do aplicativo\\ \hline
Inicio do apex&22/09/2024&19/10/2024&Marcelo Augusto&Inicio do apex                                 \\ \hline
Modificações do apex&20/10/2024&14/11/2024&Marcelo Augusto&Modificações finais do apex                                  \\ \hline
Termino de todas as atividades                  &14/11/2024                           &                &Todos os membros                 & Atividades encerradas e encaminhadas para a inserção no artigo                                 \\ \hline
                  &                            &                &                 &                                  \\ \hline
                  &                            &                &                 &                                  \\ \hline
\end{tabular}
\end{table}



\end{document}